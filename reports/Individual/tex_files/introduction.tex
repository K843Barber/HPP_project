\section{Introduction}
Lower-Upper (LU) Decomposition is the procedure whereby one factorizes a matrix A as the product of a lower triangular matrix and an upper triangular matrix. This can be viewed as a form of Gaussian elimination. This procedure can be implemented on computers. The use of LU matrices extend to solving systems of linear equations, inverting matrices and computing determinants.

There are multiple forms in which we can compute the lower and upper matrices. The most basic of these is to simply compute $A = LU$.

Another form of computing the lower and upper matrices is to include partial pivoting. The purpose of this procedure is to negate the issue if any of the diagonal terms are equal to 0. If we have a 0 term, we cannot continue the procedure to build the lower and upper matrices. A more advanced procedure of the above is to use full pivoting where we also rearrange the columns of A.

One final form is that of $A = LDU$ where D is a diagonal matrix, which allows for the diagonals of L and U to be set to 1. The benefit of this matrix, is that it can deal with rectangular matrices as well.

In the lower triangular matrix, L, all the entries above the diagonal are zero. In the upper triangular matrix, U, all the entries below the diagonal are zero. For example, the decomposition for a $3 \times 3$ matrix can take the following form:

$
\begin{bmatrix}
    a_{11} & a_{12} & a_{13}\\
    a_{21} & a_{22} & a_{23}\\
    a_{31} & a_{32} & a_{33}\\
\end{bmatrix}
$
$=$
$
\begin{bmatrix}
    l_{11} & 0 & 0\\
    l_{21} & l_{22} & 0\\
    l_{31} & l_{32} & l_{33}\\
\end{bmatrix}
$
$
\begin{bmatrix}
    u_{11} & u_{12} & u_{13}\\
    0 & u_{22} & u_{23}\\
    0 & 0 & u_{33}\\
\end{bmatrix}
$
