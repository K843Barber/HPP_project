\section{The Problem}
For this assignment, we are going to implement a program that will calculate the evolution of N particles in a gravitational simulation, whereby we are given an initial set of particles. The simulation will be done in 2 spatial dimensions with an $x$ and $y$ coordinate.

To help describe the evolution of these particles, we use Newton's law of gravitation in two dimensions, which states that the force exerted on particle $i$ by particle $j$ is given by

    $$\boldsymbol{f}_{ij} = -\frac{Gm_{i}m_{j}}{\|\boldsymbol{r}_{ij}\|^{3}}\boldsymbol{r}_{ij}.$$

where G is the gravitational constant, $m_{i}$ and $m_{j}$ are the masses of the particles, and $\boldsymbol{r}_{ij}$ is the distance vector given by $\boldsymbol{x}_{i} -\boldsymbol{x}_{j}$ with $\boldsymbol{x}_{i}$ being the position of particle $i$.

We will make use of the following force equation (Plummer Sphere Force Equation) to describe the forces on the particles. The force on particle $i$ is given by

    $$\boldsymbol{F}_{i} = -Gm_{i}\sum_{j = 0, j \neq i}^{N-1}\frac{m_{j}}{(\|\boldsymbol{r}_{ij}\| + \epsilon_{0})^{3}}\boldsymbol{r}_{ij}.$$

where $\epsilon_{0}$ is a smoothing parameter and in the computations, we used the value $\epsilon = 10^{-3}$.

To update the particle positions, we use the Euler Symplectic Time Integration method. The equations describing this are
$$
\begin{cases}
    \boldsymbol{a}_{i}^{n} = \frac{\boldsymbol{F}_{i}^{n}}{m_{i}};
    \\
    \boldsymbol{u}_{i}^{n+1} = \boldsymbol{u}_{i}^{n} + \Delta t\boldsymbol{a}_{i}^{n};
    \\
    \boldsymbol{x}_{i}^{n+1} = \boldsymbol{x}_{i}^{n} + \Delta t\boldsymbol{u}_{i}^{n+1},
\end{cases}
$$
where $\Delta t$ is the time step size, $\boldsymbol{a}_{i}$ is the acceleration of particle $i$, $\boldsymbol{u}_{i}$ is the velocity and $\boldsymbol{x}_{i}$ is the position of particle $i$. In the computations, we used $\Delta t = 10^{-5}$. The value of G depends on N and is given by $G = 100/N$.
\newpage
